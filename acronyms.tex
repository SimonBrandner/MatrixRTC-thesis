\newacronym[description={Server, který přijmá streamy od uživatelů a jejich podmnožinu posílá zpět. (Viz \ref{sfu}.)}]{sfu}{SFU}{Selective Forwarding Unit}
\newacronym[description={Server, který příjmá streamy jednotlivých uživatelů, spojuje je v jeden a posílá jej uživatelům zpět. (Viz \ref{mcu}.)}]{mcu}{MCU}{Multipoint Control Unit}
\newacronym[description={Metoda šifrování, kde třetí strany nemohou číst data,
            neboť klíče mají jen koncové strany. Příkladem je mnoho aplikací pro
            chat, ve kterých server nemůže číst zprávy uživatelů, neboť klíče
            mají jen klienti daných uživatelů
            \parencite{IBM-EndToEndEncryption}.}, name={End-to-end encryption}
            ]{e2ee}{E2EE}{end-to-end encryption}
            \newacronym[description={Peer-to-peer spojení dvou počítačů je
            takové spojení, kde k výměně dat není třeba žádný server
            \parencite{MerriamWebster-PeerToPeer}.},
            name={Peer-to-peer}]{p2p}{P2P}{peer-to-peer}
\newacronym[description={Metoda pro mapování privátních adres v podsíti routeru na jednu veřejnou adresu roueru. (Viz \ref{nat}.)}]{nat}{NAT}{Network Address Traversal}
\newacronym[description={Protokol, který umožňuje získání veřejné IP adresy routeru, za kterým se klient nachází. (Viz \ref{stun}.)}]{stun}{STUN}{Session Traversal Utilities for NAT}
\newacronym[description={Metoda, která využívá proxy (prostředníky) pro přeposílání dat. Využívá se, pokud nejde jinak obejít problémy vytvořené \gls{nat}. (Viz \ref{turn}.)}]{turn}{TURN}{Traversal Using Relays around NAT}
\newacronym[description={Protokol, který hledá nejlepší způsob, jak propojit dva počítače. (Viz \ref{ice}.)}]{ice}{ICE}{Interactive Connectivity Establishment}
\newacronym[description={Komunikační technologie sloužící pro telefonní spojení pomocí internetu \parencite{Britannica-VoIP}.}]{voip}{VoIP}{Voice over Internet Protocol}
\newacronym[description={Protokol pro přenos dat mezi počítači se zárukou doručení \parencite{Britannica-TCP}.}]{tcp}{TCP}{Transmission Control Protocol}
\newacronym[description={Protokol pro přenos dat mezi počítači bez záruky doručení \parencite{GeeksForGeeks-UDP}.}]{udp}{UDP}{User Datagram Protocol}
\newacronym[description={Protokol, který specifikuje, jak si mají dva klienti vyměňovat informace o jejich veřejných adresách, podporovaných kodecích atp. (Viz \ref{signalling}.)}]{sdp}{SDP}{Session Description Protocol}
\newacronym[description={Rozhraní, které umožňuje dvěma komponentům spolu komunikovat pomocí definovaných protokolů \parencite{AWS-WhatIsAnAPI}.}]{api}{API}{Application Programming Interface}
\newacronym[description={Protokol určený pro přenos audio a videa přes internet. (Viz \ref{rtp}.)}]{rtp}{RTP}{Real-time Transport Protocol}
\newacronym[description={Jedná se o jazyk popisující strukturu a obsah webu \parencite{MDN-Web-HTML}.}]{html}{HTML}{HyperText Markup Language}
\newacronym[description={Skupina nástrojů určená pro zjednodušení vývoje na dané
            platformě \parencite{RedHat-WhatIsAnSDK}.}]{sdk}{SDK}{Software
            Development Kit}
\newacronym[description={Protokol pro přeposílání libovolných dat přes \gls{webrtc}. (Viz \ref{sctp}.)}]{sctp}{SCTP}{Stream Control Transmission Protocol}
\newacronym[description={Formální návrh pro změnu specifikace protokolu Matrix. (Viz \ref{spec}.)}]{msc}{MSC}{Matrix Spec Change}
\newacronym[description={Protokol pro přeposílání audia a videa přes internet \parencite{RestStream-RTMP}.}]{rtmp}{RTMP}{Real Time Messaging Protocol}
\newacronym[description={Protokol, který generuje klíče pro šifrování dat
            posílaných přes protokol \gls{udp}. (Viz
            \ref{dtls}.)}]{dtls}{DTLS}{Datagram Transport Layer Security}
            \newacronym[description={Protokol využívající externí šifrovací
            klíče vygenerované \gls{dtls} k šifrování \gls{rtp} paketů. (Viz
            \ref{srtp}.)}]{srtp}{SRTP}{Secure Real-time Transport Protocol}
\newacronym[description={Protokol pro přeposílání metadat o spojení pomocí protokolu \gls{rtp}. (Viz \ref{rtcp}.)}]{rtcp}{RTCP}{RTP Control Protocol}
\newacronym[description={Tým, který se stará o specifikaci protokolu Matrix a schvaluje její změny. (Viz \ref{spec}.)}]{sct}{SCT}{Spec Core Team}
\newacronym[description={Doba před přijetím \gls{msc} do specifikace protokolu Matrix, kdy je prostor na poslední komentáře. (Viz \ref{spec}.)}]{fcp}{FCP}{Final Comment Period}
\newacronym[description={Protokol určený pro šifrování dat posílaných protokolem \gls{tcp}. \parencite{MicrosoftLearn-TransportLayerSecurityProtocol}.}]{tls}{TLS}{Transport Layer Security}
\newacronym[description={Protokol pro šifrování zpráv ve skupinách dvou a více uživatelů \parencite{MLS-MessagingLayerSecurity}. (Viz \ref{matrix-encryption}.)}]{mls}{MLS}{Messaging Layer Security}
\newacronym[description={Decentralizovaná verze šifrovacího protokolu \gls{mls}. (Viz \ref{matrix-encryption}.)}]{dmls}{DMLS}{Decentralized Messaging Layer Security}
\newacronym[description={Protokol, který specifikuje komunikaci se signalling serverem. (Viz \ref{sip}.)}]{sip}{SIP}{Session Initiation Protocol}
\newacronym[description={Běžná veřejná telefonní síť \parencite{Nextiva-PSTN}.}]{pstn}{PSTN}{Public Switched Telephone Network}
\newacronym[description={Systém, ve kterém má každá číslice přiřazenou dvojici
            tónů. Tyto tóny se pak používají pro komunikaci mezi zařízeními
            \parencite{DSTNY-DTMF}.}]{dtmf}{DTMF}{Dual-Tone Multi-Frequency}
\newacronym[description={Legislativa, která má za cíl zamezit tvoření monopolů. (Viz \ref{dma}.)}]{dma}{DMA}{Digital Markets Act}
\newacronym[description={Experimentální počítačová síť, která byla předchůdcem internetu \parencite{Britannica-ARPANET}.}]{arpanet}{ARPANET}{Advanced Research Projects Agency Network}
\newacronym[description={Aktuálně nejvyužívanější symetrická bloková šifra \parencite{Pelzl-UnderstandingCryptography}.}]{aes}{AES}{Advanced Encryption Standard}
\newacronym[description={Protokol pro provádění \gls{voip} hovorů na webu. (Viz \ref{webRTC}.)}]{webrtc}{WebRTC}{Web Real-Time Communication}
