\documentclass[aspectratio=169]{beamer}

\usepackage[czech]{babel}
\usepackage{tikz}
\usepackage{csquotes}
\usepackage{grid-system}
\usepackage{verbatim}

\usetheme[
	background=dark,
	numbering=none,
	progressbar=frametitle
]{metropolis}

\usetikzlibrary{
	positioning,
	decorations,
	tikzmark,
	tikzmark,
	fit,
	shapes.misc,
	decorations.pathreplacing,
	calc,
	matrix,
	overlay-beamer-styles
}

\setbeamercovered{transparent}

\makeatletter
\setlength{\metropolis@titleseparator@linewidth}{1.5pt}
\setlength{\metropolis@progressonsectionpage@linewidth}{1.5pt}
\setlength{\metropolis@progressinheadfoot@linewidth}{1.5pt}
\makeatother

\graphicspath{{images/}}

\title{Využití protokolu Matrix pro videohovory}
\subtitle{Ročníková práce}
\author{Šimon Brandner}
\date{6. března 2024}

\begin{document}
\maketitle
\begin{frame}{Protokoly aplikací MS Teams, WhatsApp, Discord, Zoom atp.}
    \pause
    \begin{columns}
        \begin{column}{0.5\textwidth}
            \centering
            \begin{tikzpicture}[every node/.style={draw,circle}]
                \onslide<+-> {
                    \foreach \i in {1,...,3} {
                            \node (client\i) at ({90-360/3*(\i-1)}:3.1) {Klient \i};
                        }
                }
                \onslide<+-> {
                    \node[minimum size=1.75cm] (server) at (0, 0) {Server};
                }
                \onslide<+-> {
                    \foreach \i in {1,...,3} {
                            \draw[green,<->] (server) -- (client\i);
                        }
                }
            \end{tikzpicture}
        \end{column}
        \begin{column}{0.5\textwidth}
            \begin{itemize}[<+->]
                \item Uzavřená specifikace protokolu
                \item Centralizované
                \item Poskytovatel ovládá protokol
                \item Poskytovatel ovládá data
                \item Poskytovatel ovládá uživatele
            \end{itemize}
        \end{column}
    \end{columns}
\end{frame}
\begin{frame}{Protokol Matrix}
    \pause
    \begin{columns}
        \begin{column}{0.5\textwidth}
            \centering
            \begin{tikzpicture}[every node/.style={draw,circle}]
                \onslide<+-> {
                    \foreach \i in {1,...,3} {
                            \node (client\i) at ({90-360/3*(\i-1)}:3.1) {Klient \i};
                        }
                }
                \onslide<+-> {
                    \foreach \i in {1,...,3} {
                            \node[minimum size=1.75cm] (server\i) at ({90-360/3*(\i-1)}:1.15) {Server \i};

                            \draw[green,<->] (server\i) -- (client\i);
                        }
                }
                \onslide<+-> {
                    \foreach \i in {1,...,2} {
                            \foreach \j in {\the\numexpr\i+1,...,3} {
                                    \draw[cyan,<->] (server\i) -- (server\j);
                                }
                        }
                }
            \end{tikzpicture}
        \end{column}
        \begin{column}{0.5\textwidth}
            \begin{itemize}[<+->]
                \item Federace -- decentralizace
                \item Svoboda uživatele
                \item Otevřená specifikace protokolu
                \item Interoperabilita
            \end{itemize}
        \end{column}
    \end{columns}
\end{frame}
\begin{frame}{Cíle a obsah práce}
    \pause
    \begin{itemize}[<+->]
        \item Prozkoumat protokol Matrix a jeho fungování
        \item Prozkoumat fungování aplikací pro videohovory
              \begin{itemize}
                  \item Protokoly WebRTC, STUN, TURN a ICE
                  \item Skupinové videohovory
              \end{itemize}
        \item Praktická část -- implementace break-out rooms
              \begin{itemize}
                  \item Návrh doplnění specifikace protokolu Matrix
                  \item Přidání funkce do aplikace Element Call
              \end{itemize}
    \end{itemize}
\end{frame}
\begin{frame}{Praktická část -- break-out rooms}
    \pause
    \begin{columns}[T]
        \begin{column}{0.5\textwidth}
            \begin{tikzpicture}
                \onslide<+-> {
                    \node[draw] (mainRoom) at (-2, 0) {Hlavní místnost};
                }

                \onslide<+-> {
                    \foreach \j in {1,...,3} {
                            \node[draw] (breakoutRoom\j) at (1.5, 2-\j) {Break-out místnost \j};
                            \draw[green,->] (mainRoom.east) -- (breakoutRoom\j.west);
                        }
                }
            \end{tikzpicture}
        \end{column}
        \begin{column}{0.5\textwidth}
            \begin{itemize}[<+->]
                \item Jeden hovor na místnost
                \item Break-out místnosti opravdu místnosti
                \item Kontrola nad právy v místnostech
            \end{itemize}
        \end{column}
    \end{columns}
\end{frame}
\begin{frame}{Budoucnost}
    \pause
    \begin{itemize}[<+->]
        \item Break-out rooms jako součást protokolu a aplikace Element Call
        \item Third Room -- Metaverse s použitím protokolu Matrix
        \item Digital Markets Act (DMA) a interoperabilita
    \end{itemize}
\end{frame}
\section{Děkuji za pozornost}
\begin{comment}
\begin{frame}{Protokol WebRTC}
    \centering
    \begin{tikzpicture}
        \pause
        \node[draw] (client1) at (-6, 0) {Klient 1};
        \node[draw] (client2) at (6, 0) {Klient 2};
        \pause
        \node[draw] (router1) at (-3.5, 0) {Router 1};
        \draw[green, <->] (client1) -- (router1);
        \node[draw] (router2) at (3.5, 0) {Router 2};
        \draw[green, <->] (client2) -- (router2);
        \pause
        \node[draw] (stun) at (0,1.5) {STUN server};
        \draw[yellow, <->] (router1) -- (stun);
        \draw[yellow, <->] (router2) -- (stun);
        \pause
        \node[draw, alt=<8>{orange}] (signalling) at (0,-1.5) {Signalling server};
        \draw[red, <->] (router1) -- (signalling);
        \draw[red, <->] (router2) -- (signalling);
        \pause
        \onslide<6> \draw[cyan, <->] (router1) -- (router2) node[midway, anchor=south]
        {Data};
        \pause
        \node[draw] (turn) at (0,0) {TURN server};
        \draw[cyan, <->] (router1) -- (turn) node[midway, anchor=south]
        {Data};
        \draw[cyan, <->] (router2) -- (turn) node[midway, anchor=south]
        {Data};
    \end{tikzpicture}
\end{frame}

\foreach \i in {1,...,5} {
        \newcounter{sfuEdgeCounter\i}
    }
\begin{frame}{Skupinové WebRTC}
    \begin{columns}
        \begin{column}{0.5\textwidth}
            \centering
            Full-mesh\\
            \vspace{10pt}
            \begin{tikzpicture}[every node/.style={draw,circle}]
                \onslide<+-> \foreach \i in {1,...,5} {
                        \node (\i) at ({(90-360/5*(\i-1))}:2) {\i};
                    }
                \onslide<+-> \foreach \i in {1,...,4} {
                        \foreach \j in {\the\numexpr\i+1,...,5} {
                                \draw[cyan,->] (\i) edge[bend right=5] (\j);
                                \draw[green,->] (\j) edge[bend right=5] (\i);
                                \ifnum\i=1
                                    \ifnum\j<6
                                        \onslide<+->
                                    \fi
                                \fi
                            }
                    }
            \end{tikzpicture}
        \end{column}
        \begin{column}{0.5\textwidth}
            \centering
            Selective Forwarding Unit\\
            \vspace{10pt}
            \begin{tikzpicture}[every node/.style={draw,circle}]
                \onslide<+-> \foreach \i in {1,...,5} {
                        \node (\i) at ({90-360/5*(\i-1)}:2) {\i};
                    }
                \onslide<+-> \node (SFU) at (0,0) {SFU};
                \foreach \i in {1,...,5} {
                        \setcounter{sfuEdgeCounter\i}{1}
                        \onslide<+-> \draw[green,->] (\i) edge[bend right=5] (SFU);
                        \ifnum\i>1
                            \foreach \j in {1,...,\the\numexpr\i-1} {
                                    \draw[cyan,->] (SFU) edge[bend right=5*\value{sfuEdgeCounter\j}]
                                    (\j);
                                    \stepcounter{sfuEdgeCounter\j};
                                }
                            \foreach \j in {1,...,\the\numexpr\i-1} {
                                    \draw[cyan,->] (SFU) edge[bend right=5*\value{sfuEdgeCounter\i}] (\i);
                                    \stepcounter{sfuEdgeCounter\i};
                                }
                        \fi
                    }
            \end{tikzpicture}
        \end{column}
    \end{columns}
\end{frame}
\begin{frame}{Demo}
    \centering
    \Huge Modlete se, že to bude pracuj!
\end{frame}
\end{comment}
\end{document}
