\section*{Anotace}
V posledních letech se značně rozšířilo využívání aplikací určených pro
videokonference po internetu. K jejich rozvoji výrazně přispěla pandemie. Trend
v jejich využívání však zůstává zachován a stávají se běžným nástrojem pro
komunikaci. Většina aplikací, která poskytuje tyto funkce, je ale buď
proprietární (closed-source), např. Zoom, Google Meet, MS Teams, či protokol,
který používají, není otevřený, např. Jitsi. V této práci se zaměříme na
otevřený protokol Matrix pro komunikaci v reálném čase, jehož součástí bude
skupinové volání (dále MatrixRTC). V současné době je tato podpora pro skupinové
volání v pokročilé fázi vývoje. Cílem této práce je rozebrat jednotlivé
komponenty aplikací pro skupinové volání a prozkoumat samotný protokol MatrixRTC
a jeho implementace, zhodnotit jeho použitelnost, flexibilitu, rozšiřitelnost
atp. V praktické části je poté cílem navrhnout a implementovat tzv. breakout
rooms pomocí MatrixRTC.

\section*{Klíčová slova}
Matrix, WebRTC, VoIP, otevřený software

\section*{Annotation}

\section*{Keywords}
Matrix, WebRTC, VoIP, open source
