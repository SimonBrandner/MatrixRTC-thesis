\section{Implementace break-out rooms}\label{breakoutRooms}

Break-out rooms (dále break-out místnosti) je funkce mnoha konferenčních
aplikací umožňující rozdělení velké konference do několika menších hovorů
\parencite{Zoom-EnablingMeetingBreakoutRooms,Microsoft-UseBreakoutRoomsInMSTeamsMeetings}.
Tato funkce sehrála důležitou roli např. ve školství během pandemie Covidu-19
\parencite{AhmedKhalid-TheImpactOfUtilizingBreakoutRooms,Agustina-ZoomBreakoutRoomsForStudents}
a zůstává nadále relevantní.

Praktickou částí této ročníkové práce je implementace break-out místností do
klienta Element Call, který využívá protokol Matrix pro video-hovory. Než budeme
ale samotné break-out místnosti implementovat, bude nutné navrhnout změnu Matrix
specifikace, aby případně mohli i jiní klienti využívat tuto funkci.

\subsection{Návrh na doplnění specifikace}\label{breakoutRoomsMSC}

Pro reprezentaci jednotlivých break-out místností využijeme již existující
Matrix primitivum -- místnost, každé break-out místnosti bude tedy odpovídat
jedna Matrix místnost. Pro informování ostatních uživatelů o existenci daných
break-out místností též využijme existující Matrix primitivum -- state události
\parencite{GitHub-MSC3985}:

\begin{figure}[H]
    \begin{minted}[tabsize=4,fontsize=\footnotesize]{json}
{
	"type": "m.breakout",
	"state_key": "",
	"content": {
		"m.breakout": {
			"!roomId1": {
				"via": ["example.org", "other.example.org"],
				"users": ["userId1", "userId2"]
			},
			"!roomId2": {
				"via": ["example.org", "other.example.org"]
			}
		}
	}
}
	\end{minted}
\end{figure}

Pokud chce klient vytvořit např. 2 break-out místnosti, vytvoří dané Matrix
místnosti a následně pošle state událost typu \mintinline{text}{m.breakout},
jejíž příklad můžeme vidět výše. V~poli \mintinline{text}{content} můžeme najít
pole \mintinline{text}{m.breakout}, ve kterém klíče jsou ID Matrix break-out
místností (\mintinline{text}{!roomId1} a \mintinline{text}{!roomId2}) a hodnoty
obsahující informace o těchto místnostech. V~informacích o místnosti najdeme
pole \mintinline{text}{via}, což je výčet serverů, které je možno využít k
připojení se k dané místnosti (ID místnosti nestačí pro připojení se k místnosti
\parencite{MatrixORG-Spec}). V těchto informacích můžeme též najít pole
\mintinline{text}{users}, což je výčet uživatelů, kterým je doporučeno se do
dané místnosti připojit~\parencite{GitHub-MSC3985}.

Dojde-li k přijetí této state událostí jiným klientem, měl by uživateli dát
možnost se připojit do daných místností, zároveň by ve svém uživatelském
rozhraní měl zobrazit, kteří uživatelé náleží kterým místnostem
\parencite{GitHub-MSC3985}.

Jelikož je nepraktické, aby v místnosti bylo víc událostí typu
\mintinline{text}{m.breakout}, není to povoleno. Toho docílíme tím, že jako
\mintinline{text}{state_key} budeme vždy využívat prázdný řetězec znaků (tedy
\mintinline{text}{""}) \parencite{GitHub-MSC3985} -- v Matrix místnosti vždy
existuje jen jedna validní state událost s danou dvojicí \mintinline{text}{type}
a \mintinline{text}{state_key} \parencite{MatrixORG-Spec}.

Na GitHubu v repositáři
\href{https://github.com/matrix-org/matrix-spec-proposals}{matrix-spec-proposals}
můžeme najít pull request č. 3985 --
\href{https://github.com/matrix-org/matrix-spec-proposals/pull/3985}{MSC3985},
kde je návrh změny specifikace veřejně dostupný \parencite{GitHub-MSC3985}.

\subsection{Samotná implementace}

Pro přidání funkce break-out místností do aplikace Element Call budeme muset
provést změny ve dvou repositářích --
\href{https://github.com/matrix-org/matrix-js-sdk/}{matrix-js-sdk} (Matrix
JavaScript SDK -- více v části \ref{matrix-js-sdk}) a
\href{https://github.com/vector-im/element-call/}{element-call} (samotná
aplikace Element Call). Jak Element Call tak Matrix JavaScript SDK využívají
programovací jazyk TypeScript \parencite{GitHub-MatrixJSSDK,
    GitHub-ElementCall}, což je nadstavba na JavaScriptu, která jej rozšiřuje o
statické typování \parencite{TypeScript-Homepage}.

\subsubsection{Matrix JavaScript SDK}

Změny, o kterých bude v této části práce řeč, lze nalézt v
\href{https://github.com/matrix-org/matrix-js-sdk/pull/3753/}{zde}
\parencite{GitHub-MatrixJSSDK-BreakoutRooms}.

Prvním krokem bude přidání nové veřejné metody s názvem
\mintinline{typescript}{createBreakoutRooms()} do třídy
\mintinline{typescript}{MatrixClient}, kterou najdeme v souboru
\mintinline{text}{/src/client.ts}. Tato metoda bude mít dva parametry --
\mintinline{text}{parentRoomId}, což je ID místnosti, ze které break-out
místnosti vytváříme, a \mintinline{text}{rooms}, což je seznam break-out
místností, o kterých chceme informovat další klienty v místnosti:

\begin{figure}[H]
    \begin{minted}[tabsize=4,fontsize=\footnotesize]{typescript}
export class MatrixClient extends TypedEventEmitter<
	EmittedEvents,
	ClientEventHandlerMap,
> {
	public async createBreakoutRooms(
		parentRoomId: string,
		rooms: BreakoutRoom[],
	): Promise<ISendEventResponse>;
}
	\end{minted}
\end{figure}

Metoda projde seznam \mintinline{text}{rooms} a vytvoří z něj
\mintinline{text}{content} pro \mintinline{text}{m.breakout} state událost (více
v části \ref{breakoutRoomsMSC}). Metoda
\mintinline{typescript}{createBreakoutRooms()} state událost typu
\mintinline{text}{m.breakout} následně odešle do místnosti s ID
\mintinline{text}{parentRoomId} pomocí metody
\mintinline{typescript}{this.sendStateEvent()}
\parencite{GitHub-MatrixJSSDK-BreakoutRooms}.

Dále přidáme třídu \mintinline{typescript}{BreakoutRooms}, která umožňuje
monitorovat break-out místnosti, na které odkazuje \mintinline{text}{m.breakout}
state událost v dané Matrix místnosti
\parencite{GitHub-MatrixJSSDK-BreakoutRooms}:

\begin{figure}[H]
    \begin{minted}[tabsize=4,fontsize=\footnotesize]{typescript}
export class BreakoutRooms extends TypedEventEmitter<
	BreakoutRoomsEvent,
	BreakoutRoomsEventHandlerMap,
> { /* ...implementace... */ }
	\end{minted}
\end{figure}

V konstruktoru třídy si uložíme instanci \mintinline{typescript}{room}, což je
místnost, ve které budeme monitorovat změny události typu
\mintinline{text}{m.breakout}:

\begin{figure}[H]
    \begin{minted}[tabsize=4,fontsize=\footnotesize]{typescript}
export class BreakoutRooms extends TypedEventEmitter<
	BreakoutRoomsEvent,
	BreakoutRoomsEventHandlerMap,
> {
	public constructor(private room: Room);
}
	\end{minted}
\end{figure}

Zároveň voláním metody \mintinline{typescript}{room.addListener()} docílíme
toho, že bude-li \mintinline{typescript}{room} emitovat událost typu
\mintinline{typescript}{RoomEvent.Timeline}, bude volána metoda
\mintinline{typescript}{this.onEvent()}:

\begin{figure}[H]
    \begin{minted}[tabsize=4,fontsize=\footnotesize]{typescript}
export class BreakoutRooms extends TypedEventEmitter<
	BreakoutRoomsEvent,
	BreakoutRoomsEventHandlerMap,
> {
	public constructor(private room: Room) {
		room.addListener(RoomEvent.Timeline, this.onEvent);
	}
}
	\end{minted}
\end{figure}

Posledním krokem v konstruktoru třídy
je získání aktuální Matrix state události typu \mintinline{text}{m.breakout}
pomocí metody \mintinline{typescript}{this.getBreakoutEvent()} a její zpracování
pomocí metody
\mintinline{typescript}{this.parseBreakoutEvent()}~\parencite{GitHub-MatrixJSSDK-BreakoutRooms}.

Privátní metoda \mintinline{typescript}{getBreakoutEvent()} najde Matrix událost
typu \mintinline{text}{m.breakout} v místnosti
\mintinline{typescript}{this.room} a následně ji vrátí
\parencite{GitHub-MatrixJSSDK-BreakoutRooms}:

\begin{figure}[H]
    \begin{minted}[tabsize=4,fontsize=\footnotesize]{typescript}
export class BreakoutRooms extends TypedEventEmitter<
	BreakoutRoomsEvent,
	BreakoutRoomsEventHandlerMap,
> {
	private getBreakoutEvent(): MatrixEvent | null;
}
	\end{minted}
\end{figure}

Privátní metoda \mintinline{typescript}{parseBreakoutEvent()} překonvertuje
\mintinline{text}{content} state události typu \mintinline{text}{m.breakout} na
seznam \mintinline{typescript}{BreakoutRoomWithSummary[]}, který následně vrátí:

\begin{figure}[H]
    \begin{minted}[tabsize=4,fontsize=\footnotesize]{typescript}
export class BreakoutRooms extends TypedEventEmitter<
	BreakoutRoomsEvent,
	BreakoutRoomsEventHandlerMap,
> {
	private async parseBreakoutEvent(
		event: MatrixEvent,
	): Promise<BreakoutRoomWithSummary[]>;
}
	\end{minted}
\end{figure}

Tento seznam je lépe zpracovatelný při dalším využití, neboť je obohacen o tzv.
\textit{shrnutí místnosti} (\textit{room summary}), jež obsahuje různá metadata
jako název místnosti, která nejsou obsažena v samotné
\mintinline{text}{m.breakout} state události; toto shrnutí místnosti získáme
pomocí metody \mintinline{typescript}{getRoomSummary()} na třídě
\mintinline{typescript}{MatrixClient}
\parencite{GitHub-MatrixJSSDK-BreakoutRooms}.

Privátní metoda \mintinline{typescript}{onEvent} je tzv. \textit{event handler},
který je volán, pokud je emitována událost typu
\mintinline{typescript}{RoomEvent.Timeline}. Jejím jediným parametrem je
\mintinline{typescript}{event} typu \mintinline{typescript}{MatrixEvent}. Tato
metoda nejprve pomocí metody \mintinline{typescript}{event.getType()} zjistí,
jestli je typ této Matrix události \mintinline{text}{m.breakout}, a pokud tomu
tak je, využije k získání aktuálního seznamu break-out místností metod
\mintinline{typescript}{this.getBreakoutEvent()} a
\mintinline{typescript}{this.parseBreakoutEvent()}. Pokud došlo ke změně tohoto
seznamu, je využita metoda \mintinline{typescript}{this.emit()} k emitování
informací o změně~\parencite{GitHub-MatrixJSSDK-BreakoutRooms}:

\begin{figure}[H]
    \begin{minted}[tabsize=4,fontsize=\footnotesize]{typescript}
export class BreakoutRooms extends TypedEventEmitter<
	BreakoutRoomsEvent,
	BreakoutRoomsEventHandlerMap,
> {
	private onEvent: (event: MatrixEvent) => Promise<void>;
}
	\end{minted}
\end{figure}

Veřejná metoda \mintinline{typescript}{getCurrentBreakoutRooms()} pak už jen
vrací kopii privátního seznamu
\mintinline{typescript}{this.currentBreakoutRooms}, která může být následně
využita jinde \parencite{GitHub-MatrixJSSDK-BreakoutRooms}:

\begin{figure}[H]
    \begin{minted}[tabsize=4,fontsize=\footnotesize]{typescript}
export class BreakoutRooms extends TypedEventEmitter<
	BreakoutRoomsEvent,
	BreakoutRoomsEventHandlerMap,
> {
	public getCurrentBreakoutRooms(): BreakoutRoomWithSummary[] | null;
}
	\end{minted}
\end{figure}

V konstruktoru třídy \mintinline{typescript}{Room}, kterou najdeme v souboru
\mintinline{text}{/src/models/room.ts}, vytvoříme instanci třídy
\mintinline{typescript}{BreakoutRooms}, kterou uložíme do privátního člena třídy
s názvem \mintinline{typescript}{breakoutRooms}, zároveň budeme re-emitovat
všechny události emitované \mintinline{typescript}{breakoutRooms}:

\begin{figure}[H]
    \begin{minted}[tabsize=4,fontsize=\footnotesize]{typescript}
export class Room extends ReadReceipt<RoomEmittedEvents, RoomEventHandlerMap> {
	public constructor(
		public readonly roomId: string,
		public readonly client: MatrixClient,
		public readonly myUserId: string,
		private readonly opts: IOpts = {},
	) {
		this.breakoutRooms = new BreakoutRooms(this);
		this.reEmitter.reEmit(this.breakoutRooms, [BreakoutRoomsEvent.RoomsChanged]);
	}
}
	\end{minted}
\end{figure}

Přidáme také
novou veřejnou metodu do třídy \mintinline{typescript}{Room} s názvem
\mintinline{typescript}{getBreakoutRooms()}:

\begin{figure}[H]
    \begin{minted}[tabsize=4,fontsize=\footnotesize]{typescript}
export class BreakoutRooms extends TypedEventEmitter<
	BreakoutRoomsEvent,
	BreakoutRoomsEventHandlerMap,
> {
	public getBreakoutRooms(): BreakoutRoomWithSummary[] | null;
}
	\end{minted}
\end{figure}

Tato metoda zavolá
\mintinline{typescript}{this.breakoutRooms.getCurrentBreakoutRooms()} a vrátí
její návratovou hodnotu. Díky re-emitování událostí pocházejících z
\mintinline{typescript}{breakoutRooms} a metodě
\mintinline{typescript}{getBreakoutRooms()} budou všichni s přístupem k instanci
třídy \mintinline{typescript}{Room} mít přístupné i~informace o existujících
break-out místnostech \parencite{GitHub-MatrixJSSDK-BreakoutRooms}.

\subsubsection{Element Call}

Změny, o kterých bude v této části práce řeč, lze nalézt v
\href{https://github.com/vector-im/element-call/pull/1615/}{zde}
\parencite{GitHub-ElementCall-BreakoutRooms}.

Většina změn v aplikaci Element Call spočívá převážně v úpravě a doplnění
uživatelského rozhraní o prvky sloužící k vytváření a využívání break-out
místností \parencite{GitHub-ElementCall-BreakoutRooms}. Element Call pro práci s
uživatelským rozhraním využívá knihovnu React \parencite{GitHub-ElementCall}. Ta
nám umožňuje abstrahovat části práce s jazykem \gls{html} a JavaScriptem pomocí
tzv. komponentů (a jiných struktur jako jsou např. \textit{hooks}, či
\textit{contexts}). Zjednodušeně řečeno, komponent je JavaScript funkce, jejíž
návratová hodnota bude později překonvertována do \gls{html}, které bude
zobrazeno na stránce. Komponenty mohou mít vnitřní logiku a parametry, které
ovlivní jejich návratovou hodnotu~\parencite{React-Homepage}. (Samotné detaily
změn uživatelského rozhraní zde nebudeme rozebírat, neboť nejsou příliš
relevantní k tématu protokolu Matrix atp.)

\newpage

Aplikace využívá koncové šifrování, které je založeno na jednom klíči, jenž je
sdílený mezi uživateli pomocí URL
\parencite{GitHub-ElementCall-CompleteSPAE2EEWork}. Bude tedy nutné přidat
mechanismus na sdílení těchto klíčů. Využijeme k tomu to-device messages (více v
části \ref{toDeviceMessages}). Komponent
\mintinline{typescript}{BreakoutRoomModal} obsahuje funkci
\mintinline{typescript}{onSubmit()}, která rozešle pomocí metody
\mintinline{typescript}{client.sendToDevice()} klíče k nově vytvořeným break-out
místnostem ostatním klientům\footnote{Metodu
    \mintinline{typescript}{sendToDevice()} využíváme pouze pro demonstrační účely
    -- správně bychom měli využít metodu
    \mintinline{typescript}{encryptAndSendToDevices()}, aby data byla šifrovaná.}:

\begin{figure}[H]
    \begin{minted}[tabsize=4,fontsize=\footnotesize]{typescript}
export const BreakoutRoomModal: FC<Props> = ({ client, roomId, open, onDismiss }) => {
	const onSubmit: () => Promise<void>;
};
	\end{minted}
\end{figure}

Zároveň v kontextu \mintinline{typescript}{ClientContext}, jenž je v souboru
\mintinline{text}{src/ClientContext.tsx}, přidáme nový event handler s názvem
\mintinline{typescript}{onToDevice()}, který budeme volat, přijde-li nová
to-device message:

\begin{figure}[H]
    \begin{minted}[tabsize=4,fontsize=\footnotesize]{typescript}
export const ClientProvider: FC<Props> = ({ children }) => {
	const onToDevice: (event: MatrixEvent) => void;
};
	\end{minted}
\end{figure}

Bude-li obsahovat informace o nových klíčích, uložíme je
k~dalšímu použití \parencite{GitHub-ElementCall-BreakoutRooms}.
