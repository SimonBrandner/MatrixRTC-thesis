\section{Závěr}

Cílem této práce bylo rozebrat poměrně nezmapovanou oblast -- fungování
\gls{voip} aplikací a fungování protokolu Matrix. Dalším cílem bylo vytvořit
nový návrh doplnění specifikace protokolu Matrix a implementovat break-out rooms
do Matrix JavaScript SDK a aplikace Element Call.

Na prvních několika stránkách textu jsme rozebrali fungování protokolu
\gls{webrtc}, který je základem mnoha \gls{voip} aplikací. Ukázali jsme si, z
jakých dílčích protokolů se tento protokol skládá, a vysvětlili jsme si, jak
fungují.

V další části práce jsme si ukázali, jak se využívá protokol \gls{webrtc} pro
skupinové volání -- rozebrali jsme různé modely propojení více uživatelů a
jejich výhody a nevýhody.

V následující části jsme si stručně představili některé jiné existující
\gls{voip} technologie jako Jitsi a Zoom.

Poté jsme rozebrali, jak funguje protokol Matrix. Vysvětlili jsme, jak funguje
jeho otevřená specifikace a jakým způsobem se vyvíjí. Následně jsme si prošli
jednotlivá primitiva, ze kterých je protokol Matrix složený. Podívali jsme se na
několik implementací klientů a homeserverů. Také jsme vysvětlili, jak fungují
přemostění a end-to-end šifrování.

Následně jsme probrali protokol MatrixRTC (část protokolu Matrix zaměřená na
\gls{voip}), jeho využití Matrix primitiv a samotné fungování hovorů mezi dvěma
a více uživateli. Prošli jsme též implementace klientských \gls{sdk}. V
neposlední řadě jsme si ukázali koncept foci a jejich fungování.

V praktické části práce jsme navrhli doplnění protokolu Matrix o funkci
break-out rooms, která zatím není jeho součástí, a následně tento návrh
implementovali do Matrix JavaScript SDK a aplikace Element Call.

V poslední části práce jsme zaměřili na potenciální využití MatrixRTC mimo
\gls{voip} aplikace a obecný vývoj v oblasti protokolu Matrix a \gls{voip}
aplikací.

Implementace break-out rooms je plně funkční, ale před dalšími kroky musí být
samozřejmě řádně otestována. Následně bude možné, aby změny Matrix JavaScript
SDK a Element Call byly přijaty. Na poli protokolu Matrix by dalším krokem bylo
posunutí návrhu změny specifikace do fáze \gls{fcp}.
