\section{Budoucnost}\label{future}

Je těžké předpovídat, jak se budou VoIP aplikace vyvíjet v příštích několika
letech, ale některá fakta nám mohou mírně napovědět -- na ně se v této části
podíváme.

\subsection{Využití MatrixRTC v non-VoIP aplikacích}

Přestože historicky hlavním účelem MatrixRTC je VoIP pomocí protokolu Matrix,
WebRTC umožňují přenos arbitrárních informací, je tedy možné MatrixRTC využít i
pro jiné účely.

\subsubsection{Third Room -- virtuální světy postavené na Matrixu}\label{thirdroom}

Jedním z projektů, který využívá MatrixRTC, je \href{https://thirdroom.io}{Third
    Room} \parencite{ThirdRoom-Homepage}. Jedná se o vizi otevřeného
Metaverse\footnote{
    \textit{Metaverse} je virtuální svět, ve kterém mohou lidé trávit svůj čas
    prací, zábavou, socializováním atp. Termín \textit{Metaverse} vytvořil Neal
    Stephenson ve své sci-fi knize \textit{Snow Crash}
    \parencite{TechTarget-WhatIsTheMetaverse}
}, které používá Matrix jako svou signalling vrstvu pro virtuální 3D světy.

Third Room využívá engine jménem \textit{Manifold}, který je založený na
technologiích jako je \href{https://threejs.org/}{Three.js},
\href{https://github.com/NateTheGreatt/bitECS}{bitECS}
a \href{https://rapier.rs/}{Rapier}. Jako své Matrix SDK využívá \href{https://www.npmjs.com/package/hydrogen-view-sdk}{Hydrogen SDK} \parencite{ThirdRoom-Homepage}.

\subsubsection{TheBoard -- virtuální tabule}

\href{https://github.com/toger5/TheBoard}{TheBoard} je nástroj pro spolupráci,
který umožňuje svým uživatelům kreslit na virtuální tabuli v reálném
čase~\parencite{GitHub-TheBoard}. TheBoard aktuálně MatrixRTC nevyužívá, ale je to
vhodný kandidát pro jeho využití, neboť by projekt mohl využít WebRTC datové
kanály k téměř okamžité komunikaci v reálném čase.

\subsection{Web Transport -- alternativa k WebRTC}

\textit{Web Transport} je experimentální protokol pro klient-server komunikaci
založenou na UDP. Představuje alternativu k WebRTC, neboť rovněž umožňuje
posílání médií a arbitrárních dat s nízkým zpožděním \parencite{GitHub-WebTransport-Explainer}.

\subsection{Digital Markets Act}

\textit{Digital Markets Act} (\textit{DMA}) je legislativa, jejímž cílem je
vytvořit více kompetitivní prostředí mezi tzv. \textit{gatekeepers}, což jsou
velcí poskytovatelé klíčových služeb jako je chat, VoIP atp.; příkladem je Meta,
Apple, Google a Microsoft. DMA nařizuje, aby tito poskytovatelé umožnili přístup
k jejich APIs, což by umožnilo lepší interoperabilitu mezi službami. DMA zároveň
nařizuje, že u služeb, které poskytují E2E šifrování, by ho API měla zachovávat
\parencite{Element-AGuideToNavigateTheDMA}.

Aktuálně je vývoj Matrix přemostění komplikován tím, že gatekeepers nemusí mít
veřejná a zdokumentovaná API, DMA by ale mohlo tuto situaci změnit a umožnit
snadnější vývoj přemostění, což by mohlo napomoct protokolu Matrix, aby byl
využíván větším množstvím běžných uživatelů.
