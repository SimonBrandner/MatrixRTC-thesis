\section{Úvod}

29. října 1969 se zrodil \textit{The Advanced Research Projects Agency Network}
(dále \textit{ARPANET}), předchůdce dnešního internetu. Byly poslány znaky
\uv{L} a \uv{O}. Student UCLA Charley Kline se totiž snažil napsat zprávu
\uv{LOGIN}, ale systém po zadání dvou znaků zhroutil. Přibližně po hodině se mu
konečně podařilo poslat celé slovo
\cite{PBS-SimpleHelloFirstMessageOverARPANET}. Trvalo to jen dva roky, než Ray
Tomlinson poslal první e-mail, který zněl \uv{something like QWERTYUIOP}
\cite{YahooFinance-SimpleHelloFirstMessageOverARPANET}.

V roce 1974 byl proveden první tzv. \textit{voice over internet protocol} (dále
\textit{VoIP}) hovor po ARPANETu
\cite{DigiFone-WhatYouMightNotKnowAboutTheHistoryOfVoIP}. V roce 1989 v CERNu
britský vědec Tim Berners-Lee stvořil \textit{World Wide Web} (dále
\textit{WWW}) \cite{CERN-TheBirthOfTheWeb}. O dva roky později vznikla první
VoIP aplikace \textit{Speak Freely}
\cite{DigiFone-WhatYouMightNotKnowAboutTheHistoryOfVoIP} a po dalších dvou
letech se objevila první webkamera, která snímala konvici na kávu v počítačové
laboratoři na University of Cambridge \cite{BBC-FirstWebcamMadeCoffeePotFamous}.

29. srpna 2003 byla vydaná první verze aplikace \textit{Skype}
\cite{ArsTechnica-TheStrangeStoryOfSkype} a sloveso \uv{sky\-povat} se během
dalších let běžnou součástí slovníku mnoha lidí. Často je používáno prakticky
jako synonymum pro VoIP hovory a jeho běžnost užívání ukazuje na to, že VoIP
aplikace se staly běžnou součástí našich životů.

Využití VoIP aplikací zaznamenalo další \uv{boom} v době pandemie covidu-19, kdy
bylo mnoho lidí nucena pracovat z domova, a tak se tyto nástroje staly
nepostradatelnými \cite{OnSIP-VoIPStatsTrendsCovidImpact}.

Dnes většina VoIP aplikací, jako je např. Microsoft Teams, Discord, Google Meet,
Jitsi Meet atp., využívá protokolu \textit{WebRTC} (\textit{RTC} je zkratka pro
\textit{Real Time Communication}), který se stal standardem pro komunikaci na
internetu v reálném čase \cite{LevelUp-WhatPowerMeetAndTeams,
	Discord-HowDoesItHandleMillionsOfUsers, Jitsi-Projects, WebRTCORG-Homepage}.
Těmito aplikacemi se budeme zabývat v části \ref{webRTC}. Oproti prvním VoIP
aplikacím tyto přinesly mnoho zlepšení, mimo jiné např. umožňují skupinové
hovory až tisíců uživatelů \cite{MicrosoftLearn-MSTeamsLimitsAndSpecs}. To, jak
je toho docíleno, shrneme v části \ref{connectionModels}. A samotným aplikacím
se pak budeme věnovat v~části~\ref{voipTech}.

Ale přes mnoho zlepšení, ke kterým bezpochyby došlo, však v jistých ohledech
dochází spíše ke zhoršení. Například protokoly, které používáme pro e-mailovou
komunikaci, umožňují tzv. \textit{federaci}. V tomto kontextu federace znamená,
že jednotlivé e-mail servery spolu mohou komunikovat, a je tedy možné poslat
e-mail od jednoho e-mailového poskytovatele k druhému (např. uživatelé Gmailu a
Seznamu si spolu mohou bez jakýchkoliv potíží vyměňovat e-maily)
\cite{MatrixORG-FAQ}. Podobně na tom byly kdysi i aplikace Yahoo! Messenger a
MSN Messenger \cite{BetaNews-MSYahooToLinkIMNets}, ale o většině dnešních
aplikací určených pro chat to samé říct nemůžeme. Dnes už asi nikoho nenapadne,
že by si uživatel Messengeru mohl vyměňovat zprávy s uživateli What's App
\cite{9To5Mac-InteroperabilityNightmareAndDream}.

Existují ale protokoly jako \textit{ActivityPub} \cite{W3ORG-ActivityPub},
\textit{XMPP} \cite{XMPPORG-Homepage} a \textit{Matrix}
\cite{MatrixORG-Homepage} (více v části \ref{matrix}), jejichž účelem je situaci
změnit. Právě protokol Matrix bude dalším tématem, kterým se budeme zabývat,
jelikož kromě běžných funkcí pro chat také umožňuje VoIP hovory (více v části
\ref{matrixRTC}) \cite{MatrixORG-Homepage}.

Přesto, že protokol Matrix podporuje VoIP hovory, je tato část protokolu stále
ve vývoji \cite{GitHub-MSC3401,GitHub-MSC3898} a chybí jí některé funkce, jako
jsou např. break-out rooms. V částí \ref{breakoutRooms} se pokusíme navrhnout,
jak do protokolu tuto funkci přidat a následně ji i implementujeme.

Nakonec, v části \ref{future}, se podíváme na to, jak by se situace mohla
vyvíjet v~dalších letech, a pokusíme se ukázat, že protokoly využívané pro
komunikaci v reálném čase mají mnoho dalších využití kromě VoIP hovorů.
