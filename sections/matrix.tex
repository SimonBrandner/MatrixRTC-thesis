\section{Matrix}\label{matrix}

\begin{figure}[h]
    \centering
    \begin{tikzpicture}[every node/.style={draw,circle}]
        \foreach \i in {1,...,3} {
                \node[minimum size=2.5cm] (server\i) at ({90-360/3*(\i-1)}:2) {Server \i};
                \node (client\i) at ({90-360/3*(\i-1)}:5) {Klient \i};

                \draw[red] (server\i) -- (client\i);
            }

        \foreach \i in {1,...,2} {
                \foreach \j in {\the\numexpr\i+1,...,3} {
                        \draw[blue] (server\i) -- (server\j);
                    }
            }
    \end{tikzpicture}
    \caption{Matrix federace}
    \label{federation}
\end{figure}

Protokol Matrix je otevřený standard pro bezpečnou, decentralizovanou komunikaci
v reálném čase. Protokol byl vytvořen jako způsob řešení problému tzv.
\textit{walled gardens}. Sám o sobě je protokol federovaný, tedy podobně jako
email umožňuje klientům jednoho serveru komunikovat s klienty serveru druhého
(viz obrázek č. \ref{federation}). V protokolu Matrix se těmto serverům říká
\textit{homeservery}. Od června 2019 je stabilní
\parencite{MatrixORG-FAQ,MatrixORG-Homepage}.

Je spravován neziskovou organizací jménem Matrix.org Foundation a jeho
referenční implementace používají permisivní\footnote{Licence, která nevyžaduje
    od odvozených děl, aby využívali stejnou licenci
    \parencite{MUNI-FI-VerejneLicenceVCR}.} licenci Apache \parencite{MatrixORG-Homepage}.

\subsection{Specifikace a její změny}\label{spec}

Jak \href{https://spec.matrix.org/latest/}{stabilní specifikace}, která je
vydávána jednou za čtvrt roku, tak
\href{https://spec.matrix.org/unstable/}{nestabilní specifikace} jsou volně
dostupné a kdokoliv je může implementovat.

Vzhledem k tomu, že protokol je otevřený a umožňuje všem navrhovat změny,
existuje proces proces pro návrh a schvalování těchto změn. Chce-li někdo
navrhnout změnu protokolu, může napsat tzv. \gls{msc} návrh. Jedná se o pull
request\footnote{ Pull request je způsob pro návrhu změn zdrojového kódu,
    diskuzi o nich a případné následné přijmutí těchto změn
    \parencite{GitHub-PullRequests}. } navrhující úpravu repositáře
\href{https://github.com/matrix-org/matrix-spec-proposals}{matrix-spec-proposals}.
Tento pull request obsahuje text formátovaný pomocí
\href{https://docs.github.com/en/get-started/writing-on-github/getting-started-with-writing-and-formatting-on-github/basic-writing-and-formatting-syntax}{GitHub-flavored
    Markdownu}\footnote{ Markdown je jednoduchý mark-up jazyk pro formátování textu
    \parencite{JohnGruber-Markdown}. GitHub-flavored Markdown je pak Markdown, který
    používá GitHub \parencite{GitHub-Markdown}. } a popisuje problém, který se
\gls{msc} snaží vyřešit, možnosti řešení a výhody a nevýhody daných přístupů k
řešení \parencite{MatrixORG-MSCs}.

Poté, co je pull request vytvořen, nastává prostor pro to, aby komunita
a~vývojáři \gls{msc} komentovali a aby se případné připomínky řešily. \gls{msc}
obvykle také bude zkontrolováno členy \gls{sct}, který se o specifikaci stará.
Pokud se \gls{msc} dostane do stavu, kdy nejsou vznášeny připomínky (a ty, co
byly vznešené, byly vyřešeny), může člen \gls{sct} navrhnout spuštění tzv.
\gls{fcp}. Pro to, aby \gls{fcp} začala, je třeba souhlas 75~\% \gls{sct}
\parencite{MatrixORG-MSCs}.

Když \gls{fcp} započne, je prostor 5 dnů na to, aby kdokoliv vznesl připomínky.
Jsou-li vzneseny závažné připomínky, člen \gls{sct} může \gls{fcp} zablokovat.
Proběhne-li \gls{fcp} do konce, je \gls{msc} schváleno a nastává čas pro
samotnou změnu specifikace, jedná se obvykle další pull request, který vytvoří
člen \gls{sct} a navrhuje změny repositáře
\href{https://github.com/matrix-org/matrix-spec}{matrix-spec}. Když je tento
pull request přijat, změna se stala součástí nestabilní verze specifikace a bude
v další stabilní verzi \parencite{MatrixORG-MSCs}.

\subsection{Primitiva}

Matrix má několik základních primitiv, které si zde rozebereme.

\subsubsection{Uživatelé}

Každý klient je spojený s uživatelským účtem, ty jsou identifikované pomocí
\mintinline{text}{user_id}. To má obecně následující formu:
\mintinline{text}{@localpart:domain}. Reálné uživatelské ID pak může vypadat
takto: \mintinline{text}{@alice:server.org} \parencite{MatrixORG-Spec}.

\subsubsection{Zařízení}

Zařízení mají v protokolu Matrix specifický význam -- nejedná se o fyzická
zařízení, ale jedná se o abstrakce jednotlivých přihlášených klientů. Každý
uživatel může mít několik zařízení, ze kterých využívá svůj účet. Zařízení je
obvykle identifikováno pomocí \mintinline{text}{user_id} a
\mintinline{text}{device_id} \parencite{MatrixORG-Spec}.

Zařízení mají hlavně význam pro šifrování -- viz část \ref{matrix-encryption}.

\subsubsection{Místnosti a události}

Velice zásadním primitivem v protokolu Matrix je \textit{místnost}
(\textit{room}). V základní podobě je Matrix místnost reprezentací místnosti pro
chat, ale má mnoho jiných využití, neboť se více méně jedná o decentralizovanou
databází pro JSON data. Místnost ani její obsah neexistuje na žádném jediném
homeserveru -- každý homeserver má její kopii, kterou udržuje synchronizovanou s
ostatními homeservery. Kromě místností pro chat, existují např. tzv.
\textit{prostory} (\textit{spaces}), které odkazují na jiné místnosti, a tak
umožňují vytvořit hierarchickou strukturu místností (analogií jsou např. týmy v
aplikace MS Teams) \parencite{MatrixORG-Spec}.

Místnost obsahuje tzv. \textit{události} (\textit{events}), které jsou obvykle
výsledkem akce na straně klienta (např. poslání zprávy). Události se dělí na dva
druhy -- prvním druhem jsou \textit{message události}, jedná se např. o běžné
zprávy a reakce, druhým typem jsou \textit{state události}, které reprezentují
trvalé informace, jako je např. jméno místnosti, téma místnosti, členství v
místnosti atp. \parencite{MatrixORG-Spec}.

Vzhledem k tomu, že state události můžou být měněny více uživateli, existuje
tzv. \textit{state resolution algorithm}, který umožňuje homeserverům se
shodnout na obsahu state událostí, pokud došlo k tomu, že homeservery nebyly
nějakou dobu ve spojení \parencite{MatrixORG-Spec}.

\subsubsection{To-device messages}\label{toDeviceMessages}

\textit{To-device messages} jsou dalším druhém událostí, který ale neexistuje v
místnosti, neboť je přeposílán z jednoho zařízení do druhého. Aktuálně má hlavně
dvě využití -- šifrování (více v části \ref{matrix-encryption}) a skupinové
volání (více v části \ref{matrixRTC}).

\subsection{Klienti}

V tento moment existují pravděpodobně stovky Matrix \gls{sdk} a klientů, z nichž
je většina vyvíjena komunitou. Mezi ty známé patří např.
Element~\parencite{Element-Homepage}, Hydrogen \parencite{GitHub-Hydrogen},
Nheko \parencite{GitHub-Nheko}, FluffyChat \parencite{FluffyChat-Homepage} či
Cinny \parencite{Cinny-Homepage}.

Element je všeobecný klient pro Matrix chat \parencite{Element-Homepage}, jeho
webová verze používá \gls{sdk}
\href{https://github.com/matrix-org/matrix-js-sdk/}{matrix-js-sdk}, které je
napsané v TypeScriptu \parencite{GitHub-MatrixJSSDK}. Stejné SDK používá klient
Cinny, který je specifický svým uživatelským rozhraním inspirovaným Discordem
\parencite{Cinny-Homepage,GitHub-Cinny}. Jedná se o SDK, které je jednou z
referenčních implementací protokolu Matrix \parencite{GitHub-MatrixJSSDK}.
Desktopová verze Elementu je pak Electron\footnote{ Electron je framework pro
    vytváření desktopových aplikací pomocí webových technologií
    \parencite{ElectronJS-Homepage}. } wrapper jeho webové verze
\parencite{GitHub-ElementDesktop}. Hydrogen je rovněž webová aplikace,
používající své vlastní SDK, které je z části v TypeScriptu a z části v
JavaScriptu. Toto SDK ale využívají i jiní klienti (viz \ref{thirdroom}).
Hydrogen má za cíl vyžadovat co nejnižší požadavky na hardware a měl by běžet
i~v~prohlížečích, jako je Internet Explorer\footnote{ Internet Explorer je jeden
    z nejvíce nenáviděných internetových prohlížečů, jedním z důvodů je, že
    nepodporuje mnoho moderní funkcí, ale mnoho vývojářů je s ním přesto nuceno
    pracovat \parencite{ZealousSites-WhyDoWebDevelopersHateInternetExplorer}. }
\parencite{GitHub-Hydrogen}.

Android a iOS verze Elementu využívají každá své vlastní SDK (v Kotlinu a
Objective C) \parencite{GitHub-ElementAndroid,GitHub-ElementIOS}, probíhá však
vývoj nástupců těchto aplikací jménem Element~X, které budou využívat
\href{https://github.com/matrix-org/matrix-rust-sdk}{matrix-rust-sdk}
\parencite{GitHub-ElementXAndroid,GitHub-ElementXIOS}.

Nheko je desktopový klient napsaný v jazyce C++ s pomocí knihovny Qt pro tvoření
uživatelských rozhraní \parencite{GitHub-Nheko}.

FluffyChat je klient zaměřený na jednoduchost používání, je napsaný v~jazyce
Dart, a tak běží na operačních systémech Android a iOS, ale i na webu a desktopu
\parencite{FluffyChat-Homepage,GitLab-FluffyChat}.

\subsection{Homeservery}

Tři nejvýznamnější implementace homeserverů jsou
\href{https://github.com/matrix-org/synapse/}{Synapse}, který je napsaný v
jazyce Python, jedná se o historicky první a referenční implementaci homeserveru
\parencite{GitHub-Synapse}. Druhou je
\href{https://github.com/matrix-org/dendrite/}{Dendrite}, která je napsaná v
jazyce Go a má být nástupce Synapse \parencite{GitHub-Dendrite}. Poslední
implementací je \href{https://github.com/timokoesters/conduit}{Conduit}, která
je napsaná v jazyce Rust a je spravovaná komunitou \parencite{GitHub-Conduit}.

Mezi další homeservery patří např.
\href{https://github.com/matrix-construct/construct}{Construct} v jazyce C++
\parencite{GitHub-Construct}.

\subsection{Přemostění (bridging)}

Vzhledem k malé pravděpodobnosti, že by společnosti jako Google, či Microsoft
začaly používat protokol Matrix, umožňuje tento protokol tzv. \textit{bridging}.
Jedná se o způsob, jak \uv{přemostit} uživatele z jiných platforem (jako je
např. Messenger či MS Teams) do protokolu Matrix \parencite{MatrixORG-Bridges}.

\subsection{End-to-end šifrování}\label{matrix-encryption}

Protokol Matrix podporuje \gls{e2ee} -- message události v místnostech mohou být
šifrované. Pro dosažení \gls{e2ee} používá jak symetrické, tak asymetrické
šifrování. Jednotlivé message události jsou zašifrovány pomocí symetrického
šifrování \gls{aes}. K přeposílání klíčů mezi zařízeními, které jsou v dané
místnosti, se používají to-device messages šifrované pomocí asymetrické
kryptografie nad eliptickými křivkami (využívají se především dvě knihovny --
\href{https://gitlab.matrix.org/matrix-org/olm}{olm}, která je napsané v jazyce
C \parencite{GitLab-Olm}, a
\href{https://github.com/matrix-org/vodozemac}{vodozemac}, která je napsaná v
jazyce Rust a je součástí
\href{https://github.com/matrix-org/matrix-rust-sdk}{matrix-rust-sdk})
\parencite{GitHub-MatrixRustSDK}.

Aktuálně není možné šifrovat state události, ale existují \gls{msc}, které by to
umožnily \parencite{GitHub-MSC3414}. Zároveň se pracuje na integraci protokolu
\gls{mls}, který má fungovat lépe v místnostech s velkým množstvím zařízení.
\gls{mls} je ale třeba rozšířit tak, aby podporovalo federované systémy,
rozšíření se pak nazývá \gls{dmls} \parencite{AreWeMLSYet}.
