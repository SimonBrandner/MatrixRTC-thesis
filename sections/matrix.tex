\section{Matrix}\label{matrix}

\begin{figure}[h]
	\centering
	\begin{tikzpicture}[every node/.style={draw,circle}]
		\foreach \i in {1,...,3} {
				\node[minimum size=2.5cm] (server\i) at ({90-360/3*(\i-1)}:2) {Server \i};
				\node (client\i) at ({90-360/3*(\i-1)}:5) {Klient \i};

				\draw[red] (server\i) -- (client\i);
			}

		\foreach \i in {1,...,2} {
				\foreach \j in {\the\numexpr\i+1,...,3} {
						\draw[blue] (server\i) -- (server\j);
					}
			}
	\end{tikzpicture}
	\caption{Matrix federace}
	\label{federation}
\end{figure}

Protokol Matrix je otevřený standard pro bezpečnou, decentralizovanou komunikaci
v reálném čase. Protokol byl vytvořen jako způsob řešení problému tzv.
\textit{walled gardens}. Sám o sobě je protokol federovaný, tedy podobně jako
email umožňuje klientům jednoho serveru komunikovat s klienty serveru druhého
(viz obrázek č. \ref{federation}). V protokolu Matrix se těmto serverům říká
\textit{homeservery}. Od června 2019 je je stabilní
\cite{MatrixORG-FAQ,MatrixORG-Homepage}.

Je spravován neziskovou organizací jménem Matrix.org Foundation a jeho
referenční implementace používají permisivní licenci Apache
\cite{MatrixORG-Homepage}.

\subsection{Specifikace a její změny}

Jak \href{https://spec.matrix.org/latest/}{stabilní specifikace}, která je
vydávána jednou za čtvrt roku, tak
\href{https://spec.matrix.org/unstable/}{nestabilní specifikace} jsou volně
dostupné a kdokoliv je může implementovat.

Vzhledem k tomu, že protokol je otevřený a umožňuje všem navrhovat změny,
existuje proces proces pro návrh a schvalování těchto změn. Chce-li někdo
navrhnout změnu protokolu, může napsat tzv. \textit{MSC} (Matrix Spec Change)
návrh. Jedná se o pull request\footnote{
	Pull request je způsob pro návrhu změn
	zdrojového kódu, diskuzi o nich a případné následné přijmutí těchto změn
	\cite{GitHub-PullRequests}.
}
navrhující úpravu repositáře
\href{https://github.com/matrix-org/matrix-spec-proposals}{matrix-spec-proposals}.
Tento pull request obsahuje text formátovaný pomocí
\href{https://docs.github.com/en/get-started/writing-on-github/getting-started-with-writing-and-formatting-on-github/basic-writing-and-formatting-syntax}{GitHub-flavored Markdownu}\footnote{
	Markdown je jednoduchý mark-up jazyk pro formátování textu
	\cite{JohnGruber-Markdown}. GitHub-flavored Markdown je pak Markdown, který
	používá GitHub \cite{GitHub-Markdown}.
} a popisuje problém, který se MSC snaží vyřešit, možnosti řešení a výhody a
nevýhody daných přístupů k řešení \cite{MatrixORG-MSCs}.

Poté, co je pull request vytvořen, nastává prostor pro to, aby komunita
a~vývojáři MSC komentovali a aby se případné připomínky řešily. MSC obvykle také
bude zkontrolováno členy \textit{Spec Core Týmu} (\textit{SCT}), který se o
specifikaci stará. Pokud se MSC dostane do stavu, kdy nejsou vznášeny připomínky
(a ty, co byly vznešené, byly vyřešeny), může člen SCT navrhnout spuštění tzv.
\textit{Final Comment Period} (\textit{FCP}). Pro to, aby FCP začala, je třeba
souhlas 75~\% SCT \cite{MatrixORG-MSCs}.

Když FCP započne, je prostor 5 dnů na to, aby kdokoliv vznesl připomínky.
Jsou-li vzneseny závažné připomínky, člen SCT může FCP zablokovat. Proběhne-li
FCP do konce, je MSC schváleno a nastává čas pro samotnou změnu specifikace,
jedná se obvykle další pull request, který vytvoří člen SCT a navrhuje změny
repositáře \href{https://github.com/matrix-org/matrix-spec}{matrix-spec}. Když
je tento pull request přijat, změna se stala součástí nestabilní verze
specifikace a bude v další stabilní verzi \cite{MatrixORG-MSCs}.

\subsection{Primitiva}

Matrix má několik základních primitiv, které si zde rozebereme.

\subsubsection{Uživatelé}

Každý klient je spojený s uživatelským účtem, ty jsou identifikované pomocí
\mintinline{text}{user_id}. To má obecně následující formu:
\mintinline{text}{@localpart:domain}. Reálné uživatelské ID pak může vypadat
takto: \mintinline{text}{@alice:server.org} \cite{MatrixORG-Spec}.

\subsubsection{Zařízení}

Zařízení mají v protokolu Matrix specifický význam -- nejedná se o fyzická
zařízení, ale jedná se o abstrakce jednotlivých přihlášených klientů. Každý
uživatel může mít několik zařízení, ze kterých využívá svůj účet. Zařízení je
obvykle identifikováno pomocí \mintinline{text}{user_id} a
\mintinline{text}{device_id} \cite{MatrixORG-Spec}.

Zařízení mají hlavně význam pro šifrování -- viz část \ref{matrix-encryption}.

\subsubsection{Místnosti a události}

Velice zásadním primitivem v protokolu Matrix je \textit{místnost}
(\textit{room}). V základní podobě je Matrix místnost reprezentací místnosti
pro chat, ale má mnoho jiných využití, neboť se více méně jedná o
decentralizovanou databází pro JSON data. Místnost ani její obsah neexistuje na
žádném jediném homeserveru -- každý homeserver má její kopii, kterou udržuje
synchronizovanou s ostatními homeservery. Kromě místností pro chat, existují
např. tzv. \textit{prostory} (\textit{spaces}), které odkazují na jiné
místnosti, a tak umožňují vytvořit hierarchickou strukturu místností (analogií
jsou např. týmy v aplikace MS Teams) \cite{MatrixORG-Spec}.

Místnost obsahuje tzv. \textit{události} (\textit{events}), které jsou obvykle
výsledkem akce na straně klienta (např. poslání zprávy). Události se dělí na dva
druhy -- prvním druhem jsou \textit{message události}, jedná se např. o běžné
zprávy a reakce, druhým typem jsou \textit{state události}, které reprezentují
trvalé informace, jako je např. jméno místnosti, téma místnosti, členství v
místnosti atp. \cite{MatrixORG-Spec}.

Vzhledem k tomu, že state události můžou být měněny více uživateli, existuje tzv.
\textit{state resolution algorithm}, který umožňuje homeserverům se shodnout na
obsahu state událostí, pokud došlo k tomu, že homeservery nebyly nějakou dobu ve
spojení \cite{MatrixORG-Spec}.

\subsubsection{To-device messages}

\textit{To-device messages} jsou dalším druhém událostí, který ale neexistuje v
místnosti, neboť je přeposílán z jednoho zařízení do druhého. Aktuálně má hlavně
dvě využití -- šifrování (více v části \ref{matrix-encryption}) a skupinové
volání (více v části \ref{matrixRTC}).

\subsection{Klienti}

V tento moment existují pravděpodobně stovky Matrix SDKs\footnote{
	\textit{Software Development Kit} je skupina nástrojů určená pro zjednodušení
	vývoj na dané platformě \cite{RedHat-WhatIsAnSDK}.
} a klientů, z nichž je většina vyvíjena komunitou. Mezi ty známé patří např.
Element~\cite{Element-Homepage}, Hydrogen \cite{GitHub-Hydrogen}, Nheko
\cite{GitHub-Nheko}, FluffyChat \cite{FluffyChat-Homepage} či Cinny
\cite{Cinny-Homepage}.

Element je všeobecný klient pro Matrix chat \cite{Element-Homepage}, jeho webová
verze používá SDK
\href{https://github.com/matrix-org/matrix-js-sdk/}{matrix-js-sdk}, které je
napsané v TypeScriptu \cite{GitHub-MatrixJSSDK}. Stejné SDK používá klient
Cinny, který je specifický svým uživatelským rozhraním inspirovaným Discordem
\cite{Cinny-Homepage,GitHub-Cinny}. Jedná se o SDK, které je jednou z
referenčních implementací protokolu Matrix \cite{GitHub-MatrixJSSDK}. Desktopová
verze Elementu je pak Electron\footnote{
	Electron je framework pro vytváření desktopových aplikací pomocí webových
	technologií \cite{ElectronJS-Homepage}.
} wrapper jeho webové verze \cite{GitHub-ElementDesktop}. Hydrogen je rovněž
webová aplikace, používající své vlastní SDK, které je z části v TypeScriptu a
z části v JavaScriptu, toto SDK ale využívají i jiní klienti (viz
\ref{thirdroom}). Hydrogen má za cíl vyžadovat co nejnižší požadavky na hardware a
měl by běžet i~v~prohlížečích, jako je Internet Explorer\footnote{
	Internet Explorer je jeden z nejvíce nenáviděných internetových prohlížečů,
	jedním z důvodů je, že nepodporuje mnoho moderní funkcí, ale mnoho vývojářů je s
	ním přesto nuceno pracovat
	\cite{ZealousSites-WhyDoWebDevelopersHateInternetExplorer}.
} \cite{GitHub-Hydrogen}.

Android a iOS verze Elementu využívají každá své vlastní SDK (v Kotlinu a
Objective C) \cite{GitHub-ElementAndroid,GitHub-ElementIOS}, probíhá však vývoj
nástupců těchto aplikací jménem Element~X, které budou využívat
\href{https://github.com/matrix-org/matrix-rust-sdk}{matrix-rust-sdk}
\cite{GitHub-ElementXAndroid,GitHub-ElementXIOS}.

Nheko je desktopový klient napsaný v jazyce C++ s pomocí knihovny Qt pro tvoření
uživatelských rozhraní \cite{GitHub-Nheko}.

FluffyChat je klient zaměřený na jednoduchost používání, je napsaný v~jazyce
Dart, a tak běží na operačních systémech Android a iOS, ale i na webu a
desktopu \cite{FluffyChat-Homepage,GitLab-FluffyChat}.

\subsection{Homeservery}

Tři nejvýznamnější implementace homeserverů jsou
\href{https://github.com/matrix-org/synapse/}{Synapse}, který je napsaný v
jazyce Python, jedná se o historicky první a referenční implementaci homeserveru
\cite{GitHub-Synapse}. Druhou je
\href{https://github.com/matrix-org/dendrite/}{Dendrite}, která je napsaná v
jazyce Go a má být nástupce Synapse \cite{GitHub-Dendrite}. Poslední implementací
\href{https://github.com/timokoesters/conduit}{Conduit}, která je napsaná v
jazyce Rust a je spravovaná komunitou \cite{GitHub-Conduit}.

Mezi další homeservery patří např.
\href{https://github.com/matrix-construct/construct}{Construct} v jazyce C++
\cite{GitHub-Construct}.

\subsection{Přemostění (bridging)}

Vzhledem k tomu k malé pravděpodobnosti, že by společnosti jako Google, či
Microsoft začaly používat protokol Matrix, umožňuje tento protokol tzv.
\textit{bridging}. Jedná se o způsob, jak \uv{přemostit} uživatele z jiných
platforem (jako je např. Messenger či MS Teams) do protokolu Matrix
\cite{MatrixORG-Bridges}.

\subsection{End-to-end šifrování}\label{matrix-encryption}

Protokol Matrix podporuje E2EE -- message události v místnostech mohou být
šifrované. Pro dosažení E2EE používá jak symetrické, tak asymetrické šifrování.
Jednotlivé message události jsou zašifrovány pomocí symetrického šifrování AES.
K přeposílání klíčů mezi zařízeními, které jsou v dané místnosti, se používají
to-device messages šifrované pomocí asymetrické kryptografie nad
eliptickými křivkami (využívají se především dvě knihovny --
\href{https://gitlab.matrix.org/matrix-org/olm}{olm}, která je napsané v jazyce
C \cite{GitLab-Olm}, a
\href{https://github.com/matrix-org/vodozemac}{vodozemac}, která je napsaná v
jazyce Rust a je součástí
\href{https://github.com/matrix-org/matrix-rust-sdk}{matrix-rust-sdk})
\cite{GitHub-MatrixRustSDK}.

Aktuálně není možné šifrovat state události, ale existují MSCs, které by to
umožnily \cite{GitHub-MSC3414}. Zároveň se pracuje na integraci protokolu
\textit{Messaging Layer Security} (dále \textit{MLS}), který má fungovat lépe v
místnostech s velkým množstvím zařízení. MLS je ale třeba rozšířit tak, aby
podporovalo federované systémy, rozšíření se pak nazývá \textit{Decentralized
	Message Layer Security} \cite{AreWeMLSYet}.
