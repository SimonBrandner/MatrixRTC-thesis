\section{Další VoIP technologie}\label{voipTech}

MatrixRTC a \gls{webrtc} samozřejmě nejsou jediné \gls{voip} technologie --
v~této části se na tyto \gls{voip} technologie podíváme.

\subsection{Protokol SIP}\label{sip}

\gls{sip} je protokol, který specifikuje komunikaci se signalling serverem,
umožňuje oznámit ostatním klientům, že je klient aktivní, volat ostatní klienty,
ukončení hovoru atp \parencite{ExtraHop-SIPOverview}.

\subsection{LiveKit}

\textit{LiveKit} je open-source projekt, který by měl zjednodušit práci vývojářů
s \gls{webrtc}. Jedná se o \gls{sfu} (viz \ref{connectionModels}) a kolekci
klient \gls{sdk} \parencite{LiveKit-Homepage,GitHub-LiveKit}.

Jeho \gls{sfu} využívá Pion a jako vrstva pro signalling se využívá
\textit{WebSockets} \parencite{GitHub-LiveKit}.

\subsection{Protokol aplikace Zoom}

Zoom je konferenční software, který stojí na protokolech SIP, \gls{rtmp} a
\gls{webrtc}. SIP se využívá na signalling, \gls{rtmp} na samotný přenos médií a
\gls{webrtc} \gls{api} pro získání streamu dat z kamery či obrazovky. Ve
webových prohlížečích používá pro přenos médií WebSockets, což je zvláštní
rozhodnutí vzhledem k tomu, že WebSockets používají
\gls{tcp}~\parencite{WebRTCHacks-HowZoomAvoidWebRTC,EBODigital-HowDoesZoomWork,
    Zoom-Homepage}.

\subsection{Protokol aplikace Discord}

Discord je aplikace na chat a \gls{voip} hovory
\parencite{Discord-WhatIsDiscord}. Používá protokol \gls{webrtc} jako základ své
technologie, neboť je podporován na celé škále prohlížečů a operačních systémů.
V prohlížečích Discord využívá jejich nativní implementace \gls{webrtc}, ale
na~desktopu, Androidu a iOS využívají nativní knihovnu pro \gls{webrtc} napsanou
v C++; některé funkce tedy fungují lépe např. v desktopové aplikaci než v
prohlížeči \parencite{Discord-HowDoesItHandleMillionsOfUsers}.

Nativní knihovna v C++ umožňuje Discordu větší kontrolu nad \gls{webrtc}, což
dává prostor pro různá zjednodušení. Vzhledem k tomu, že v případě Discordu
klient vždy volá server, je možné vynechat \gls{ice}. Discord také používá
Salsa20 šifrování, protože je rychlejší než \gls{dtls} a \gls{srtp}. Zároveň
šetří data tím, že neposílá žádné pakety, pokud uživatel zrovna nemluví
\parencite{Discord-HowDoesItHandleMillionsOfUsers}.

Signalling server používá WebSockets a je napsaný v jazyce Elixir
\parencite{Discord-HowDoesItHandleMillionsOfUsers}.

\subsection{Protokol aplikace Jitsi}

\textit{Jitsi} je kolekce open-source projektů, které poskytují \gls{voip}
služby, které jsou vhodné pro self-hosting \parencite{Jitsi-Docs-Introduction}.
Jako signalling vrstvu používá protokol XMPP \parencite{XMPPORG-WebRTC}.
