\section{Další VoIP technologie}\label{voipTech}

MatrixRTC a WebRTC samozřejmě nejsou jediné VoIP technologie -- v~této části se
na tyto VoIP technologie podíváme.

\subsection{Protokol SIP}

\textit{Session Initiation Protocol} (\textit{SIP}) je protokol, který
specifikuje komunikaci se signalling serverem, umožňuje oznámit ostatním
klientům, že je klient aktivní, volat ostatní klienty, ukončení hovoru atp \parencite{ExtraHop-SIPOverview}.

\subsection{LiveKit}

\textit{LiveKit} je open-source projekt, který by měl zjednodušit práci vývojářů
s WebRTC. Jedná se o SFU (viz \ref{connectionModels}) a kolekci klient SDKs
\parencite{LiveKit-Homepage,GitHub-LiveKit}.

Jeho SFU využívá Pion, jako signalling vrstva se využívá \textit{WebSockets} \parencite{GitHub-LiveKit}.

\subsection{Protokol aplikace Zoom}

Zoom je konferenční software, který stojí na protokolech SIP, \textit{Real Time
    Messaging Protocol} (\textit{RTMP}) a WebRTC. SIP se využívá na signalling,
RTMP na samotný přenos médií a WebRTC API pro získání streamu dat z kamery
či obrazovky. Ve webových prohlížečích používá pro přenos médií WebSockets,
což je zvláštní rozhodnutí vzhledem k tomu, že WebSockets používají TCP~\parencite{WebRTCHacks-HowZoomAvoidWebRTC,EBODigital-HowDoesZoomWork,
    Zoom-Homepage}.

\subsection{Protokol aplikace Discord}

Discord je aplikace na chat a VoIP hovory \parencite{Discord-WhatIsDiscord}. Používá
protokol WebRTC jako základ své technologie, neboť je podporován na celé škále
prohlížečů a operačních systémů. V prohlížečích Discord využívá jejich nativní
implementace WebRTC, ale na desktopu, Androidu a iOS využívají nativní knihovnu
pro WebRTC napsanou v C++; některé funkce tedy fungují lépe např. v desktopové
aplikaci než v prohlížeči \parencite{Discord-HowDoesItHandleMillionsOfUsers}.

Nativní knihovna v C++ umožňuje Discordu větší kontrolu nad WebRTC, což dává
prostor pro různá zjednodušení. Vzhledem k tomu, že v případě Discordu klient
vždy volá server, je možné vynechat ICE. Discord také používá Salsa20 šifrování,
protože je rychlejší než DTLS a SRTP. Zároveň šetří data tím, že neposílá žádné
pakety, pokud uživatel zrovna nemluví
\parencite{Discord-HowDoesItHandleMillionsOfUsers}.

Signalling server používá WebSockets a je napsaný v jazyce Elixir
\parencite{Discord-HowDoesItHandleMillionsOfUsers}.

\subsection{Protokol aplikace Jitsi}

\textit{Jitsi} je kolekce open-source projektů, které poskytují VoIP služby,
které jsou vhodné pro self-hosting \parencite{Jitsi-Docs-Introduction}. Jako
signalling vrstvu používá protokol XMPP \parencite{XMPPORG-WebRTC}.
