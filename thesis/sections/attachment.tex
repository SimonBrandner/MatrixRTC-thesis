\section{Příloha -- zdrojový kód}\label{attachment}

Zdrojový kód aplikace Element Call a Matrix JS SDK se nachází \iftoggle{soc}{v
    přiloženém ZIP souboru \mintinline{text}{SourceCode}.}{na přiloženém flash
    disku ve složce \mintinline{text}{RocnikovaPrace}.}

Pro zobrazení změn je potřeba mít nainstalovaný program \mintinline{text}{git}
pro distribuovanou správu verzí. Pro zobrazení změn v repozitáři
\mintinline{text}{matrix-js-sdk}, pak lze použít následující příkaz:
\begin{minted}[tabsize=4,fontsize=\footnotesize]{bash}
git -C ./matrix-js-sdk diff develop
\end{minted}

a k zobrazení změn v repozitáři \mintinline{text}{element-call}, lze použít příkaz:
\begin{minted}[tabsize=4,fontsize=\footnotesize]{bash}
git -C ./element-call diff livekit
\end{minted}

Pro spuštění aplikace Element Call je potřeba mít nainstalovaný \mintinline{text}{node.js} a
\mintinline{text}{yarn}. Když jsou oba programy nainstalované, je možné spustit
Element Call pomocí následujících příkazů:

\begin{minted}[tabsize=4,fontsize=\footnotesize]{bash}
cd element-call
yarn dev
\end{minted}

Web server by měl poté běžet a Element Call by měl být dostupný na následující
adrese: \href{https://localhost:3000}{https://localhost:3000}.
