\section{Úvod}

29. října 1969 se zrodil \gls{arpanet}, předchůdce dnešního internetu. Byly
poslány znaky \uv{L} a \uv{O}. Student UCLA Charley Kline se totiž snažil napsat
zprávu \uv{LOGIN}, ale systém se po zadání dvou znaků zhroutil. Přibližně po
hodině se mu konečně podařilo poslat celé slovo
\parencite{PBS-SimpleHelloFirstMessageOverARPANET}. Trvalo to jen dva roky, než
Ray Tomlinson poslal první e-mail, který zněl \uv{something like QWERTYUIOP}
\parencite{YahooFinance-SimpleHelloFirstMessageOverARPANET}.

V roce 1974 byl proveden první tzv. \gls{voip} hovor po síti \gls{arpanet}
\parencite{DigiFone-WhatYouMightNotKnowAboutTheHistoryOfVoIP}. V roce 1989 v
CERNu britský vědec Tim Berners-Lee stvořil World Wide Web (WWW)
\parencite{CERN-TheBirthOfTheWeb}. O dva roky později vznikla první \gls{voip}
aplikace \textit{Speak Freely}
\parencite{DigiFone-WhatYouMightNotKnowAboutTheHistoryOfVoIP} a po dalších dvou
letech se objevila první webkamera, která snímala konvici na kávu v počítačové
laboratoři na University of Cambridge
\parencite{BBC-FirstWebcamMadeCoffeePotFamous}.

29. srpna 2003 byla vydaná první verze aplikace \textit{Skype}
\parencite{ArsTechnica-TheStrangeStoryOfSkype} a sloveso \uv{sky\-povat} se
během dalších let běžnou součástí slovníku mnoha lidí. Často je používáno
prakticky jako synonymum pro \gls{voip} hovory a jeho běžnost užívání ukazuje na
to, že \gls{voip} aplikace se staly běžnou součástí našich životů.

Využití \gls{voip} aplikací zaznamenalo další \uv{boom} v době pandemie
covidu-19, kdy bylo mnoho lidí nuceno pracovat z domova, a tak se tyto nástroje
staly nepostradatelnými \parencite{OnSIP-VoIPStatsTrendsCovidImpact}.

Dnes většina \gls{voip} aplikací, jako je např. Microsoft Teams, Discord, Google
Meet, Jitsi Meet atp., využívá protokolu \gls{webrtc}, který se stal standardem
pro komunikaci na internetu v reálném čase
\parencite{LevelUp-WhatPowerMeetAndTeams,
    Discord-HowDoesItHandleMillionsOfUsers, Jitsi-Projects, WebRTCORG-Homepage}.
Těmito aplikacemi se budeme zabývat v části \ref{webRTC}. Výše uvedený software
oproti prvním \gls{voip} aplikacím (jako byla Speak Freely) přinesl mnoho
zlepšení, mimo jiné např. umožňuje skupinové hovory až tisíců uživatelů
\parencite{MicrosoftLearn-MSTeamsLimitsAndSpecs}. To, jak je toho docíleno,
shrneme v části \ref{connectionModels}. A samotným aplikacím se pak budeme
věnovat v~části~\ref{voipTech}.

Přes mnoho zlepšení, ke kterým bezpochyby došlo, však v jistých ohledech dochází
spíše ke zhoršení. Například protokoly, které používáme pro e-mailovou
komunikaci, umožňují tzv. \textit{federaci}. V tomto kontextu federace znamená,
že jednotlivé e-mail servery spolu mohou komunikovat, a je tedy možné poslat
e-mail od jednoho e-mailového poskytovatele k~druhému (např. uživatelé Gmailu a
Seznamu si spolu mohou bez jakýchkoliv potíží vyměňovat e-maily)
\parencite{MatrixORG-FAQ}. Podobně na tom byly kdysi i~aplikace Yahoo! Messenger
a MSN Messenger \parencite{BetaNews-MSYahooToLinkIMNets}. O většině dnešních
aplikací určených pro chat však to samé říct nemůžeme. Dnes už asi nikoho
nenapadne, že by si uživatel Messengeru mohl vyměňovat zprávy s uživateli What's
App \parencite{9To5Mac-InteroperabilityNightmareAndDream}.

Existují ale protokoly jako \textit{ActivityPub} \parencite{W3ORG-ActivityPub},
\textit{XMPP} \parencite{XMPPORG-Homepage} a \textit{Matrix}
\parencite{MatrixORG-Homepage} (více v~části \ref{matrix}), jejichž účelem je
situaci změnit. Právě protokol Matrix bude dalším tématem, kterým se budeme
zabývat, jelikož kromě běžných funkcí pro chat také umožňuje \gls{voip} hovory
pomocí \gls{webrtc} (více v~části \ref{matrixRTC})
\parencite{MatrixORG-Homepage,MatrixORG-Spec}.

Přestože protokol Matrix podporuje \gls{voip} hovory, je tato část protokolu
stále ve vývoji~\parencite{GitHub-MSC3401,GitHub-MSC3898} a chybí jí některé
funkce, jako jsou např. break-out rooms. V částí \ref{breakoutRooms} se pokusíme
navrhnout, jak do protokolu tuto funkci přidat a následně ji i implementujeme.

Nakonec, v části \ref{future}, se podíváme na to, jak by se situace mohla
vyvíjet v~dalších letech, a pokusíme se ukázat, že protokoly využívané pro
komunikaci v reálném čase mají mnoho dalších využití kromě \gls{voip} hovorů.

Aktuální verze práce (s jejím zdrojovým kódem) je veřejně dostupná na GitHubu.
Lze ji nalézt na adrese
\href{https://github.com/SimonBrandner/RocnikovaPrace-MatrixRTC}{github.com/SimonBrandner/RocnikovaPrace-MatrixRTC}.
